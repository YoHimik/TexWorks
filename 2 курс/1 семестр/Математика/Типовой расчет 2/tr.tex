\documentclass[12pt]{article}
\usepackage{mathtext} 
\usepackage{amsmath}

\usepackage[english, russian]{babel}
\usepackage[TS1, T2A]{fontenc}
\usepackage[utf8]{inputenc}
\usepackage{pscyr}
\usepackage{enumitem,kantlipsum}
\usepackage{tikz}
\tikzset{
  jumpdot/.style={mark=*,solid},
  excl/.append style={jumpdot,fill=white},
  incl/.append style={jumpdot,fill=black},
}
\usepackage{pgfplots}
\pgfplotsset{compat=1.9}

\usepackage[left=2cm,right=2cm, top=1cm,bottom=1.5cm,bindingoffset=0cm]{geometry}

\begin{document}
	\pagestyle{empty}
	
	\begin{center}
		\normalsize
		\textbf{Федеральное государственное автономное образовательное учреждение высшего образования}

		\small
		\medskip 
		\textbf{САНКТ-ПЕТЕРБУРГСКИЙ НАЦИОНАЛЬНЫЙ ИССЛЕДОВАТЕЛЬСКИЙ  УНИВЕРСИТЕТ ИНФОРМАЦИОННЫХ ТЕХНОЛОГИЙ, МЕХАНИКИ И ОПТИКИ}

		\medskip 
		\textbf{ФАКУЛЬТЕТ ПРОГРАММНОЙ ИНЖЕНЕРИИ И КОМПЬЮТЕРНОЙ ТЕХНИКИ}	
	\bigskip\bigskip\bigskip\bigskip\bigskip\bigskip\bigskip\bigskip\bigskip\bigskip\bigskip\bigskip	
		\par\medskip\par\smallskip\par\smallskip
		\Large 
		\textbf{Типовой расчет по математике} 

		\textbf{Модуль 6}

		\large
		\par\bigskip
		\textbf{«Ряды»}
		\par\bigskip\par\bigskip\par\bigskip\par\bigskip\par\bigskip\par\bigskip
		\par\bigskip\par\bigskip\par\bigskip\par\bigskip\par\bigskip\par\bigskip
		\par\bigskip\par\bigskip\par\bigskip\par\bigskip\par\bigskip\par\bigskip
		\normalsize
		\begin{tabular}{lllll}
							\hspace{170pt}	 							& \hspace{80pt}	&	Выполнил:								&\\
																		&			&	Студент группы P3255					&\\
																		& 			&	Федюкович С. А. \_\_\_\_\_\_\_\_\_\_\_\_\_\_	&\\
																		&			&	Вариант 26									&\\
																		&			&										&\\
		\end{tabular}
		\par\bigskip\par\bigskip\par\bigskip                                                  
		\par\bigskip \par\bigskip
		\par\bigskip\par\bigskip\par\bigskip\par\bigskip\par\bigskip\par\bigskip\par\bigskip\par\bigskip
		
		Санкт-Петербург
		\par\bigskip
		2019
	\end{center}
	\newpage
	\pagestyle{plain}
	\setcounter{page}{1}

	\section*{Задача 1}	
	\subsection*{Условие}
	
	Исследовать сходимость числовых рядов:  

	\begin{enumerate}
		\item Cходится или расходится положительный ряд $ \sum_{n=1}^{\infty} \frac{5^n + 2^n}{10^n} $
		\item Cходится или расходится знакопеременный ряд  $ \sum_{n=2}^{\infty} \frac{(-1)^n}{n\sqrt{\ln{n}}} $; если ряд сходится, установить характер сходимости.
	\end{enumerate}

	\subsection*{Решение}
	\begin{enumerate}[wide, labelwidth=!, labelindent=0pt]
		\item Применим интегральный признак Коши. Для этого введем функцию $ f(x) = \frac{5^x + 2^x}{10^x} $, удовлетворяющую условиям интегрального признака и исследуем несобственный интеграл от этой функции:		
		$$ \int \limits_1^{+ \infty} f(x) \, dx = \int \limits_1^{+ \infty} \frac{5^x + 2^x}{10^x} \, dx = \lim_{b \to + \infty} \int \limits_1^b ( \frac{1}{2^x} + \frac{1}{5^x}) \, dx = \lim_{b \to + \infty} \Big[ - \frac{1}{ 2^x \ln{2} } - \frac{1}{ 5^x \ln{5} } \Big] \Big|_0^b = $$
		$$ = \lim_{b \to + \infty} ( - \frac{1}{ 2^b \ln{2} } - \frac{1}{ 5^b \ln{5} } - (- \frac{1}{ 2 \ln{2} } - \frac{1}{ 5 \ln{5} } ) )  =  \frac{1}{ 2 \ln{2} } + \frac{1}{ 5 \ln{5} }$$

		Несобственный интеграл сходится, значит, по интегральному признаку Коши сходится исследуемый ряд. 

		\item Данный ряд является знакочередующимся, поэтому применим признак Лейбница. Последовательность абсолютных величин членов данного ряда является убывающей:
		$$ \frac{1}{n\sqrt{\ln{n}}} > \frac{1}{(n+1)\sqrt{\ln{(n+1)}}} $$		
		Также:
		$$ \lim_{n \to \infty} \frac{1}{n\sqrt{\ln{n}}} = 0 $$
		Значит по признаку Лейбница данный ряд сходится.

		Исследуем характер сходимости. Для этого рассмотрим положительный ряд $ \sum_{n=2}^{\infty} \frac{1}{n\sqrt{\ln{n}}} $, снова применив интегральный признак Коши. Для этого введем функцию $ f(x) = \frac{1}{x\sqrt{\ln{x}}} $, удовлетворяющую условиям интегрального признака и исследуем несобственный интеграл от этой функции:
		$$ \int \limits_2^{+ \infty} f(x) \, dx = \int \limits_2^{+ \infty} \frac{1}{x\sqrt{\ln{x}}} \, dx = \lim_{b \to + \infty} \int \limits_2^b \frac{d(\ln{n})}{\sqrt{\ln{x}}} = \lim_{b \to + \infty} 2\sqrt{\ln{n}} \Big|_0^b = \lim_{b \to + \infty} ( 2\sqrt{\ln{b}} - 2\sqrt{\ln{2}} ) = \infty $$

		Несобственный интеграл расходится, значит, по интегральному признаку Коши расходится исследуемый ряд. Следовательно сходящийся знакочередующийся ряд $ \sum_{n=2}^{\infty} \frac{(-1)^n}{n\sqrt{\ln{n}}} $ сходится условно. 		
	\end{enumerate}

	\hspace{200pt}\textbf{Ответ:} 1) ряд сходится; 2) ряд сходится условно.

	\newpage

	\section*{Задача 2}	
	\subsection*{Условие}
	
	Найти область сходимости степенного ряда:
	$$ \sum_{n=1}^{\infty} \frac{(x + 1)^{2n} \ln{n}}{\sqrt{n^2 + 1}} $$

	\subsection*{Решение}

	Согласно обобщенному признаку Даламбера, интервал сходимости находится из условия:
	$$ \lim_{n \to \infty} \frac{|u_{n+1}(x)|}{|u_{n}(x)|} < 1 ,$$
	где $ u_{n}(x) $ --- общий член ряда. В нашем случае:
	$$ u_{n}(x) = \frac{(x + 1)^{2n} \ln{n}}{\sqrt{n^2 + 1}},  u_{n+1}(x) = \frac{(x + 1)^{2(n+1)} \ln{(n+1)}}{\sqrt{(n+1)^2 + 1}} .$$	
	Тогда: 
	$$ \lim_{n \to \infty} \frac{|u_{n+1}(x)|}{|u_{n}(x)|} = \lim_{n \to \infty} \Big| \frac{\frac{(x + 1)^{2(n+1)} \ln{(n+1)}}{\sqrt{(n+1)^2 + 1}}}{\frac{(x + 1)^{2n} \ln{n}}{\sqrt{n^2 + 1}}} \Big| = \lim_{n \to \infty} \Big| \frac{(x + 1)^2 \ln{(n+1)} \sqrt{n^2 + 1}}{\ln{n} \sqrt{(n+1)^2 + 1}} \Big| = $$
	$$ = (x + 1)^2 \lim_{n \to \infty} \Big| \sqrt{\frac{n^2 + 1}{(n+1)^2 + 1}} \cdot \frac{\ln{(n+1)} }{\ln{n} } \Big| = (x + 1)^2 $$
	Перейдём к неравенству:
	$$ \lim_{n \to \infty} \frac{|u_{n+1}(x)|}{|u_{n}(x)|} < 1 \Leftrightarrow (x + 1)^2 < 1 \Leftrightarrow |x + 1| < 1 \Leftrightarrow - 1 < x + 1 < 1 \Leftrightarrow - 2 < x < 0  $$
	В граничных точках $ x = - 2 $ и $ x = 0 $  получим положительный ряд:
	$$ \sum_{n=1}^{\infty} \frac{(1)^{2n} \ln{n}}{\sqrt{n^2 + 1}} = \sum_{n=1}^{\infty} \frac{ \ln{n}}{\sqrt{n^2 + 1}} $$
	Для получения оценки общего члена данного ряда рассмотрим неравенства:
	\begin{equation*}
		\begin{cases}
			\ln{n} > 1\\
			\sqrt{n^2 + 1} > n\\
		\end{cases}
		\text{при } n > 2 \Rightarrow \frac{ \ln{n}}{\sqrt{n^2 + 1}} > \frac{1}{n} \text{ при } n > 2 
	\end{equation*}
	Значит, при $ n > 2 $ каждый член положительного ряда больше соответствующего члена гармонического ряда, который расходится. По первому признаку сравнения в граничных точках $ x = - 2 $ и $ x = 0 $ исследуемый степенной ряд расходится, т. е. область его абсолютной сходимости --- интервал $ (-2;0) $.

	\hspace{150pt}\textbf{Ответ:} область абсолютной сходимости ряда --- интервал $ (-2;0) $.

	\newpage

	\section*{Задача 3}	
	\subsection*{Условие}
	
	Разложить функции в степенные ряды по степеням $ x $, используя стандартные разложения. Указать интервалы их сходимости.

	\begin{enumerate}
		\item $ f(x) = \frac{x}{10} \ln{(1 + 10x)} $
		\item $ f(x) = \frac{6}{3x^2 + 2} $
	\end{enumerate}

	\subsection*{Решение}

	\begin{enumerate}[wide, labelwidth=!, labelindent=0pt]
		\item Разложим функцию, используя разложение $ f(x) = \ln{(1 + x)} $, затем каждый член полученного ряда почленно умножим на $ \frac{x}{10} $:
		$$ f(x) = \frac{x}{10} \cdot \ln{(1 + 10x)} = \frac{x}{10} \cdot \sum_{n=0}^{\infty} \frac{(-1)^{n} (10x)^{n+1}}{n + 1} = \sum_{n=0}^{\infty} \frac{(-1)^{n} (10x)^{n+1} x}{10(n + 1)} $$

		Степенной ряд $ f(x) = \ln{(1 + x)} $ сходится абсолютно при условии $ -1 < x \leq 1  $, поэтому область сходимости полученного ряда будем искать из неравенства $ -1 < 10x \leq 1  $, откуда следует $ -1/10 < x \leq 1/10  $. Таким образом, область абсолютной сходимости представляет собой интервал $ (-1/10;1/10] $.

		\item Представим данную функцию в виде:
		$$ f(x) = \frac{6}{3x^2 + 2} = 3 \cdot \frac{1}{1 + 3/2x^2} $$

		Разложим функцию, используя разложение $ f(x) = \frac{1}{1 + x} $, затем каждый член полученного ряда почленно умножим на $ 3 $:
		$$ f(x) = 3 \cdot \frac{1}{1 + 3/2x^2} = 3 \cdot \sum_{n=0}^{\infty} (-1)^{n} (3/2x^2)^n  = \sum_{n=0}^{\infty} (-1)^{n} (9/2x^2)^n $$

		Степенной ряд  $ f(x) = \frac{1}{1 + x} $ сходится абсолютно при условии $ -1 < x < 1  $, поэтому область сходимости полученного ряда будем искать из неравенства $ -1 < 3/2x^2 < 1  $, откуда следует $ -\sqrt{2/3} < x < \sqrt{2/3} $. Таким образом, область абсолютной сходимости представляет собой интервал $ (-\sqrt{2/3};\sqrt{2/3}) $.
	\end{enumerate}

	\hspace{150pt}\textbf{Ответ:} 1) $ \sum_{n=0}^{\infty} \frac{(-1)^{n} (10x)^{n+1} x}{10(n + 1)} $ на $ (-1/10;1/10] $;

	\hspace{6.6cm} 2) $ \sum_{n=0}^{\infty} (-1)^{n} (9/2x^2)^n $ на $ (-\sqrt{2/3};\sqrt{2/3}) $.

	\newpage

	\section*{Задача 4}	
	\subsection*{Условие}

	Разложить функцию $ f(x) =  x^2 - 2 $ при $ 0 \leq x \leq 2  $ в ряд Фурье на промежутке $ [-2;2] $. Доопределить её на промежутке $ [-2;0) $ нечетным образом.	Сделать схематический чертеж графика функции $ f(x) $ на промежутке $ [-2;2] $ и графика суммы $ S(x) $ ряда Фурье этой функции.

	\subsection*{Решение}

	Разложение в ряд Фурье нечетной функции содержит только синусы и имеет вид:
	$$ S(x) = \sum_{n=1}^\infty b_n \sin{\frac{n \pi x}{p}}, \text{ где } b_n = \frac{2}{p} \int_0^p f(x) \sin{\frac{n \pi x}{p}} \, dx .$$
	В нашем случае $ p = 2 $. Найдём $ b_n $:
	$$ b_n = \frac{2}{p} \int_0^p f(x) \sin{\frac{n \pi x}{p}} \, dx = \int_0^2 (x^2 - 2) \sin{\frac{n \pi x}{2}} \, dx = \int_0^2 (x^2 - 2 ) \Big( - \frac{2}{\pi n} \Big) \, d \Big( \cos{\frac{n \pi x}{2}} \Big) = $$
	$$ \frac{2(x^2 - 2)}{\pi n} \cos{\frac{n \pi x}{2}} \Big|_0^2 + \frac{4}{\pi n} \int_0^2 x \cos{\frac{n \pi x}{2}} \, dx = \frac{4 (1 +\cos{\pi n})}{\pi n} - \frac{16 \cos{((\pi n x)/2)} + 8 n \pi x \sin{((n \pi x)/2)}}{n^3 \pi^3} \Big|_0^2 = $$
	$$ = - \frac{8 (\pi n \cos{((\pi n)/2)} - 2 \sin{((\pi n)/2)})^2}{n^3 \pi^3}$$
	Ряд Фурье для функции $ f(x) $ имеет вид: 
	$$ S(x) = - \frac{8}{\pi^3} \sum_{n=1}^\infty \frac{(\pi n \cos{((\pi n)/2)} - 2 \sin{((\pi n)/2)})^2}{n^3 } \sin{\frac{n \pi x}{2}} $$
	Графики функции $ f(x) $ и суммы $ S(x) $ ряда Фурье изображены ниже
	\begin{center}
		\begin{tikzpicture}
			\begin{axis}[
			width=175,
			height=200,
			axis lines = center,
			xlabel = {$x$},
			ylabel = {$y$},
			xmin=-3,
			xmax=3,
			ymin=-3,
			ymax=3,
			xtick={-2,0, 2},
			ytick={-2,0,2}
			]
			\addplot[black, domain=0:2,]{x^2-2};
			\addplot[black, domain=-2:0,]{-x^2+2};
			\addplot[excl] coordinates {(0,2)};
			\end{axis}
		\end{tikzpicture}
		\begin{tikzpicture}
			\begin{axis}[
			width=300,
			height=200,
			axis lines = center,
			xlabel = {$x$},
			ylabel = {$y$},
			xmin=-5.5,
			xmax=5.5,
			ymin=-3,
			ymax=3,
			xtick={-4,-2,0, 2,4},
			ytick={-2,0,2}
			]
			\addplot[black, domain=0:2,]{x^2-2};
			\addplot[black, domain=-2:0,]{-x^2+2};
			\addplot[excl] coordinates {(0,2)};
			\addplot[excl] coordinates {(0,-2)};
			\addplot[excl] coordinates {(2,-2)};
			\addplot[excl] coordinates {(2, 2)};
			\addplot[excl] coordinates {(-2,2)};
			\addplot[excl] coordinates {(-2,-2)};
			\addplot[black, domain=4:5.9,]{(x-4)^2-2};
			\addplot[black, domain=2:4,]{-(x-4)^2+2};
			\addplot[excl] coordinates {(4,2)};
			\addplot[excl] coordinates {(4,-2)};
			\addplot[black, domain=-4:-2,]{(x+4)^2-2};
			\addplot[black, domain=-5.9:-4,]{-(x+4)^2+2};
			\addplot[excl] coordinates {(-4,2)};
			\addplot[excl] coordinates {(-4,-2)};
			\addplot[incl] coordinates {(-4,0)};
			\addplot[incl] coordinates {(-2,0)};
			\addplot[incl] coordinates {(0,0)};
			\addplot[incl] coordinates {(2,0)};
			\addplot[incl] coordinates {(4,0)};
			\end{axis}
		\end{tikzpicture}
	\end{center}

	\hspace{150pt}\textbf{Ответ:} $ S(x) = - \frac{8}{\pi^3} \sum_{n=1}^\infty \frac{(\pi n \cos{((\pi n)/2)} - 2 \sin{((\pi n)/2)})^2}{n^3 } \sin{\frac{n \pi x}{2}} $.
\end{document}