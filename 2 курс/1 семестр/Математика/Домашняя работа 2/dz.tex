\documentclass[12pt]{article}
\usepackage{mathtext} 
\usepackage{amsmath}

\usepackage[english, russian]{babel}
\usepackage[TS1, T2A]{fontenc}
\usepackage[utf8]{inputenc}
\usepackage{pscyr}
\usepackage{enumitem,kantlipsum}
% \usepackage{tikz}
% \tikzset{
%   jumpdot/.style={mark=*,solid},
%   excl/.append style={jumpdot,fill=white},
%   incl/.append style={jumpdot,fill=black},
% }
% \usepackage{pgfplots}
% \pgfplotsset{compat=1.9}

\usepackage[left=2cm,right=2cm, top=1cm,bottom=1.5cm,bindingoffset=0cm]{geometry}

\begin{document}
	\pagestyle{empty}
	
	\begin{center}
		\normalsize
		\textbf{Федеральное государственное автономное образовательное учреждение высшего образования}

		\small
		\medskip 
		\textbf{САНКТ-ПЕТЕРБУРГСКИЙ НАЦИОНАЛЬНЫЙ ИССЛЕДОВАТЕЛЬСКИЙ  УНИВЕРСИТЕТ ИНФОРМАЦИОННЫХ ТЕХНОЛОГИЙ, МЕХАНИКИ И ОПТИКИ}

		\medskip 
		\textbf{ФАКУЛЬТЕТ ПРОГРАММНОЙ ИНЖЕНЕРИИ И КОМПЬЮТЕРНОЙ ТЕХНИКИ}	
	\bigskip\bigskip\bigskip\bigskip\bigskip\bigskip\bigskip\bigskip\bigskip\bigskip\bigskip\bigskip	
		\par\medskip\par\smallskip\par\smallskip
		\Large 
		\textbf{Домашняя работа по математике} 

		\textbf{Модуль 6}

		\large
		\par\bigskip
		\textbf{«Ряды»}
		\par\bigskip\par\bigskip\par\bigskip\par\bigskip\par\bigskip\par\bigskip
		\par\bigskip\par\bigskip\par\bigskip\par\bigskip\par\bigskip\par\bigskip
		\par\bigskip\par\bigskip\par\bigskip\par\bigskip\par\bigskip\par\bigskip
		\normalsize
		\begin{tabular}{lllll}
							\hspace{170pt}	 							& \hspace{80pt}	&	Выполнил:								&\\
																		&			&	Студент группы P3255					&\\
																		& 			&	Федюкович С. А. \_\_\_\_\_\_\_\_\_\_\_\_\_\_	&\\
																		&			&	Вариант 26									&\\
																		&			&										&\\
		\end{tabular}
		\par\bigskip\par\bigskip\par\bigskip                                                  
		\par\bigskip \par\bigskip
		\par\bigskip\par\bigskip\par\bigskip\par\bigskip\par\bigskip\par\bigskip\par\bigskip\par\bigskip
		
		Санкт-Петербург
		\par\bigskip
		2019
	\end{center}
	\newpage
	\pagestyle{plain}
	\setcounter{page}{1}

	\section*{Задача 1}	
	\subsection*{Условие}
	
	Исследовать сходимость числовых рядов:  

	\begin{enumerate}
		\item $ \sum_{n=1}^{\infty} (-1)^n \ln{\frac{2n + 3}{2n + 1}} $
		\item $ \sum_{n=1}^{\infty} \frac{2^{n^2}}{(3n)!} $
		\item $ \sum_{n=1}^{\infty} arctg \, \frac{n}{3^n + n^2} $
		\item $ \sum_{n=1}^{\infty} \frac{(-1)^n}{n \ln^{3/4} n} $
	\end{enumerate}

	\subsection*{Решение}
	\begin{enumerate}[wide, labelwidth=!, labelindent=0pt]
		\item Данный ряд является знакочередующимся, поэтому применим признак Лейбница. Последовательность абсолютных величин членов данного ряда является убывающей:
		$$ \ln{\frac{2(n+1) + 3}{2(n+1) + 1}} = \ln{\frac{2n + 5}{2n + 3}} < \ln{\frac{2n + 3}{2n + 1}} $$		
		Также:
		$$ \lim_{n \to \infty} \ln{\frac{2n + 3}{2n + 1}} = \lim_{n \to \infty} \ln{\frac{2 + 3/n}{2 + 1/n}}  = \ln{1} = 0 $$
		Значит по признаку Лейбница данный ряд сходится. Исследуем характер сходимости. Для этого с помощью предельного признака сравнения исследуем положительный ряд $ \sum_{n=1}^{\infty} \ln{\frac{2n + 3}{2n + 1}} $. Для этого рассмотрим данный ряд:
		$$ \sum_{n=1}^\infty \frac{2}{2n+1} $$
		Он расходится по интегральному признаку Коши:
		$$ \int_1^\infty \frac{2}{2n+1} \, dn = \lim_{b \to \infty} \int_1^b \frac{1}{2n+1} \, d(2n+1) = \lim_{b \to \infty} \ln{(2n+1)} \Big|_1^b = \lim_{b \to \infty} (\ln{(2b+1)} - \ln{3}) = \infty  $$
		Сравним эти ряды:
		$$ \lim_{n \to \infty} \frac{\ln{\frac{2n + 3}{2n + 1}}}{\frac{2}{2n+1}} = \lim_{n \to \infty} \frac{\ln{( 1 + \frac{2}{2n + 1}})}{\frac{2}{2n+1}} = 1 $$
		Получено конечное число, отличное от нуля, значит, исследуемый ряд расходится вместе с рядом $ \sum_{n=1}^\infty \frac{2}{2n+1} $. Следовательно данный ряд сходится условно. \\
		В ходе вычисления последнего предела был использован замечательный предел:
		$$ \lim_{\alpha \to 0} \frac{\ln{(1 + \alpha)}}{\alpha} = 1 ,$$
		где в качестве бесконечно малой величины выступает $ \alpha = \frac{2}{2n+1} $ 
		
		\newpage
		\item Применим признак Даламбера:		
		$$ \lim_{n \to \infty} \frac{\frac{2^{(n+1)^2}}{(3(n+1))!}}{\frac{2^{n^2}}{(3n)!}} = \lim_{n \to \infty} \frac{2^{(2n+1)}}{(3n+1)(3n+2)(3n+3)} = \lim_{n \to \infty} \frac{2^{(2n+1)}}{27 n^3 + 54 n^2+  33 n  + 6} = \frac{\infty}{\infty} $$
		Дабы избавиться от неопределенности применим правило Лопиталя три раза: 
		$$ \lim_{n \to \infty} \frac{(2^{(2n+1)})'''}{(27 n^3 + 54 n^2+  33 n  + 6)'''} = \lim_{n \to \infty} \frac{(2^{(2n+2)} \ln{2})''}{(81 n^2 + 108 n + 33 )''} = $$
		$$ = \lim_{n \to \infty} \frac{(2^{(2n+3)} \ln^2{2})'}{(162 n + 108)'} = \lim_{n \to \infty} \frac{2^{(2n+4)} \ln^3{2}}{162 } = \infty $$
		Предел равен бесконечности, следовательно ряд расходится.

		\item Применим признак Даламбера:		
		$$ \lim_{n \to \infty} \frac{arctg \, \frac{n+1}{3^{(n+1)} + (n+1)^2}}{arctg \, \frac{n}{3^n + n^2}} $$
		Так как 
		$$ \lim_{n \to \infty} \frac{n+1}{3^{(n+1)} + (n+1)^2} = 0 ;\, \lim_{n \to \infty} \frac{n}{3^n + n^2} = 0 $$
		(очевидно, используя правило Лопиталя один раз), то воспользуемся эквивалентным преобразованием $ arctg (\alpha(x)) \approx \alpha(x) $:
		$$ \lim_{n \to \infty} \frac{arctg \, \frac{n+1}{3^{(n+1)} + (n+1)^2}}{arctg \, \frac{n}{3^n + n^2}} = \lim_{n \to \infty} \frac{\frac{n+1}{3^{(n+1)} + (n+1)^2}}{ \frac{n}{3^n + n^2}} = \lim_{n \to \infty} \frac{(n+1)(3^n + n^2)}{n(3^{(n+1)} + (n+1)^2)} = $$
		$$ = \lim_{n \to \infty} \frac{n+1}{n} \cdot \lim_{n \to \infty} \frac{3^n + n^2}{3^{(n+1)} + (n+1)^2} = \lim_{n \to \infty} \frac{3^n + n^2}{3^{(n+1)} + n^2 + 2n + 1} $$
		Дабы избавиться от неопределенности применим правило Лопиталя два раза: 
		$$ \lim_{n \to \infty} \frac{(3^n + n^2)''}{(3^{(n+1)} + n^2 + 2n + 1)''} = \lim_{n \to \infty} \frac{(3^n \ln{3} + 2n)'}{(3^{(n+1)} \ln{3} + 2n + 2 )'} = $$
		$$ = \lim_{n \to \infty} \frac{3^n \ln^2{3} + 2}{3^{(n+1)} \ln^2{3} + 2 } = \lim_{n \to \infty} \frac{ \frac{1}{3} + \frac{2}{3^{(n+1)} \ln^2{3}} }{ 1 + \frac{2}{3^{(n+1)} \ln^2{3}} } = \frac{1}{3} $$
		Предел равен $ \frac{1}{3} $, следовательно ряд сходится.

		\newpage
		\item Данный ряд является знакочередующимся, поэтому применим признак Лейбница. Последовательность абсолютных величин членов данного ряда является убывающей:
		$$ \frac{1}{n \ln^{3/4} n} > \frac{1}{(n+1) \ln^{3/4} (n+1)} $$		
		Также:
		$$ \lim_{n \to \infty} \frac{1}{n \ln^{3/4} n} =  0 $$
		Значит по признаку Лейбница данный ряд сходится. Исследуем характер сходимости. Для этого рассмотрим положительный ряд $ \sum_{n=1}^{\infty} \frac{1}{n \ln^{3/4} n} $, применив интегральный признак Коши. Для этого введем функцию $ f(x) = \frac{1}{x \ln^{3/4} x} $, удовлетворяющую условиям интегрального признака и исследуем несобственный интеграл от этой функции:
		$$ \int \limits_1^{+ \infty} f(x) \, dx = \int \limits_1^{+ \infty} \frac{1}{x \ln^{3/4} x}  \, dx = \lim_{b \to + \infty} \int \limits_1^b \frac{1}{x \ln^{3/4} x} = \lim_{b \to + \infty} \int \limits_1^b \ln^{-3/4} x \, d\ln{x} = $$
		$$ = \lim_{b \to + \infty} (4 (\ln{x})^{(1/4)}) \Big|_1^b = \lim_{b \to + \infty} (4 (\ln{b})^{(1/4)} - 4 (\ln{1})^{(1/4)}) =  \infty $$
	\end{enumerate}
	\textbf{Ответ:} 1) ряд сходится условно; 2) ряд расходится; 3) ряд сходится; 4) ряд сходится условно.

	\newpage

	\section*{Задача 2}	
	\subsection*{Условие}
	
	Найти область сходимости степенного ряда:
	$$ \sum_{n=1}^{\infty} \frac{n^5}{(x-1)^{4n}} $$

	\subsection*{Решение}

	Согласно обобщенному признаку Даламбера, интервал сходимости находится из условия:
	$$ \lim_{n \to \infty} \frac{|u_{n+1}(x)|}{|u_{n}(x)|} < 1 ,$$
	где $ u_{n}(x) $ --- общий член ряда. В нашем случае:
	$$ u_{n}(x) = \frac{n^5}{(x-1)^{4n}},  u_{n+1}(x) = \frac{(n+1)^5}{(x-1)^{4(n+1)}}.$$	
	Тогда: 
	$$ \lim_{n \to \infty} \frac{|u_{n+1}(x)|}{|u_{n}(x)|} = \lim_{n \to \infty} \Big| \frac{\frac{n^5}{(x-1)^{4n}}}{\frac{(n+1)^5}{(x-1)^{4(n+1)}}} \Big| = \lim_{n \to \infty} \Big| \frac{n^5(x-1)^{4(n+1)}}{(n+1)^5 (x-1)^{4n}} \Big| = $$
	$$ = (x - 1)^4 \lim_{n \to \infty} \Big| \frac{n^5 }{(n+1)^5} \Big| = (x - 1)^4 $$
	Перейдём к неравенству:
	$$ \lim_{n \to \infty} \frac{|u_{n+1}(x)|}{|u_{n}(x)|} < 1 \Leftrightarrow (x - 1)^4 < 1 \Leftrightarrow |x - 1| < 1 \Leftrightarrow - 1 < x - 1 < 1 \Leftrightarrow 0 < x < 2  $$
	В граничных точках $ x = 0 $ и $ x = 2 $  получим положительный ряд:
	$$ \sum_{n=1}^{\infty} \frac{n^5}{(1)^{4n}} = \sum_{n=1}^{\infty} n^5 $$
	Для получения оценки общего члена данного ряда рассмотрим неравенства:
	$$ n^5 > \frac{1}{n} \text{, при } n > 1 $$
	Значит, при $ n > 1 $ каждый член положительного ряда больше соответствующего члена гармонического ряда, который расходится. По первому признаку сравнения в граничных точках $ x = 0 $ и $ x = 2 $ исследуемый степенной ряд расходится, т. е. область его абсолютной сходимости --- интервал $ (0;2) $.

	\hspace{150pt}\textbf{Ответ:} область абсолютной сходимости ряда --- интервал $ (0;2) $.

	\newpage

	\section*{Задача 3}	
	\subsection*{Условие}
	
	Найти три первых отличных от нуля члена разложения функции в ряд Маклорена:
	$$ f(x) = arctg(1-2x) $$
	\subsection*{Решение}

	Ряд маклорена имеет вид:
	$$ f(x) = \sum_{n=0}^\infty \frac{f^{(n)}(0)}{n!} x^n ,$$
	где $ f^{(n)}(0) $ --- значение $ n $-ой производной в нуле $ (f^{(0)}(0)= f(x) )$. Для заданной функции имеем: $ f(0) = arctg(1) = \frac{\pi}{4} $. Теперь найдём последовательно столько производных, сколько потребуется, чтобы три из них были отличны от нуля в точке $ x = 0 $.
	$$ f'(x) = \frac{1}{- 2 x^2 + 2 x -1 }, \, f'(0) = - 1 $$
	$$ f''(x) = \frac{-2 + 4 x}{(1 + 2 x - 2 x^2)^2}, \, f''(0) = - 2 $$
	$$ f'''(x) = \frac{4 (1 - 6 x + 6 x^2)}{(-1 + 2 x - 2 x^2)^3}, \, f'''(0) = - 4 $$

	\hspace{200pt}\textbf{Ответ:} $ arctg(1-2x) = - x - x^2 - 2/3 x^3 + ...  $

	\newpage

	\section*{Задача 4}	
	\subsection*{Условие}

	Построить ряд Тейлора данной функции в окрестности точки $ x_0 $, используя стандартные разложения Маклорена основных элементарных функций. Указать область, в которой разложение справедливо:
	\begin{enumerate}
		\item $ f(x) = x^2 \cos{(x^3 + \pi / 4)}, \, x_0 = 0; $
		\item $ f(x) = (x^2 - 3x + 2)^{-1}, \, x_0 = -3 $
	\end{enumerate}

	\subsection*{Решение}

	\begin{enumerate}[wide, labelwidth=!, labelindent=0pt]
		\item Представим данную функцию в виде:
		$$ f(x) = x^2 \cos{(x^3 + \pi / 4)} = x^2(\cos{x^3}\cos{\pi/4} - \sin{x^3}\sin{\pi/4}) = \frac{x^2}{\sqrt{2}}(\cos{x^3} - \sin{x^3}) $$	
		Тогда разложим функцию, используя разложения $ f(x) = \sin{x} $ и $ f(x) = \cos{x} $, затем почленно вычтем два ряда и каждый член полученного ряда почленно умножим на $ \frac{x^2}{\sqrt{2}} $:
		$$ f(x) = \frac{x^2}{\sqrt{2}} \Big(\sum_{n=0}^\infty \frac{(-1)^n (x^3)^{2n}}{(2n)!} - \sum_{n=0}^\infty \frac{(-1)^n (x^3)^{(2n+1)}}{(2n+1)!} \Big) = $$
		$$ = \frac{x^2}{\sqrt{2}} \Big(\sum_{n=0}^\infty \frac{(-1)^n (2n + 1 - x^3) x^{6n} }{(2n + 1)!} \Big) = \sum_{n=0}^\infty \frac{(-1)^n (2n + 1 - x^3) x^{(6n + 2)} }{\sqrt{2}(2n + 1)!} $$
		Степенные ряды $ f(x) = \sin{x} $ и $ f(x) = \cos{x} $ сходятся абсолютно при условии\\ $ - \infty < x < + \infty  $, поэтому область сходимости полученного ряда будем искать из неравенства $ - \infty < x^3 < + \infty  $, откуда следует $ - \infty < x < + \infty  $. Таким образом, область абсолютной сходимости представляет собой интервал $ (- \infty;+ \infty) $.
		\item Представим данную функцию в виде:
		$$ f(x) = (x^2 - 3x + 2)^{-1} = \frac{1}{x-2} - \frac{1}{x-1} $$
		Введём обозначение $ x - x_0 = x + 3 = -t $, откуда $ x = - t - 3 $. Тогда:
		$$ \frac{1}{x-2} - \frac{1}{x-1} = \frac{1}{- t - 5} - \frac{1}{- t - 4} = -\frac{1}{5} \cdot \frac{1}{1 + t/5} + \frac{1}{4} \cdot \frac{1}{1 + t/4} $$
		Каждое из слагаемых $ (1 + t/5)^{-1} $ и $ (1 + t/4)^{-1} $ разложим, используя разложение $ f(x) = (1 + x)^{-1} $ , затем почленно домножим суммы на $ -\frac{1}{5} $ и $ \frac{1}{4} $ и сложим ряды:
		$$ f(x) = -\frac{1}{5} (\sum_{n=0}^\infty (-1)^n (t/5)^n ) + \frac{1}{4} (\sum_{n=0}^\infty (-1)^n (t/4)^n ) = \sum_{n=0}^\infty - \frac{(-t/5)^n }{5} + \sum_{n=0}^\infty \frac{(-t/4)^n }{4} = $$
		$$ = \sum_{n=0}^\infty \frac{5(-t/4)^n - 4(-t/5)^n }{20} = \sum_{n=0}^\infty \frac{5(-1/4)^n - 4(-1/5)^n }{20} t^n, \, t = - x - 3 $$
		Степенной ряд $ f(x) = (1 + x)^{-1} $ сходится абсолютно при условии $ - 1 < x < 1  $, поэтому область сходимости полученного ряда будем искать из неравенств $ - 1 < t/4 < 1 $ и $ - 1 < t/5 < 1 $, откуда следует $ - 1 < (- x - 3)/4 < 1 \Leftrightarrow - 7 < x <1$ и $ - 1 < (- x - 3)/5 < 1 \Leftrightarrow - 8 < x <2$. Таким образом, областью абсолютной сходимости будет интервал $ (- 7; 1) $.		
	\end{enumerate}
	\textbf{Ответ:} 1) $ \sum_{n=0}^\infty \frac{(-1)^n (2n + 1 - x^3) x^{(6n + 2)} }{\sqrt{2}(2n + 1)!} $ на $ (- \infty;+ \infty) $; 2) $ \sum_{n=0}^\infty \frac{5(-1/4)^n - 4(-1/5)^n }{20} (- x - 3)^n $ на $ (- 7; 1) $.	
	\newpage
	\section*{Задача 5}	
	\subsection*{Условие}

	Вычислить интеграл с точностью $ 0,001 $:
	$$ \int_0^{0,5} \frac{1}{\sqrt[3]{1+x^3}} \, dx$$

	\subsection*{Решение}
	
	Используя стандартный ряд Маклорена для функции $ f(x) = (1+x)^\alpha $ будем иметь:
	$$ (1+x^3)^{-1/3} = 1 + \sum_{n=1}^\infty \frac{-1/3(-1/3 - 1)...(-1/3 - n + 1)x^{3n}}{n!} $$
	Интегрируя этот ряд почленно, получим:
	$$ \int_0^{0,5} \frac{1}{\sqrt[3]{1+x^3}} \, dx = \Big(x + \sum_{n=1}^\infty \frac{-1/3(-1/3 - 1)...(-1/3 - n + 1)x^{3n+1}}{n!(3n+1)} \Big) \Big|_0^{0,5} = $$
	$$ = 0,5 + \sum_{n=1}^\infty \frac{-1/3(-1/3 - 1)...(-1/3 - n + 1)(0,5)^{3n+1}}{n!(3n+1)} $$
	Полученный ряд является знакочередующимся. Отсюда, на основании признака Лейбница, следует, что абсолютная величина погрешности, возникающей при замене суммы ряда $ n $-ой частичной суммой, не превосходит модуля первого отброшенного члена. Вычисляя последовательно слагаемые полученного ряда видим, что модуль второго члена:
	$$ |a_2| = \Big| \frac{1}{1792\cdot2,25} \Big| < 0,001 $$
	Следовательно, в качестве нужного нам приближения достаточно взять:
	$$ S_1 = 0,5 - \frac{1}{192} \approx 0,494 $$
	\hspace{350pt}\textbf{Ответ:} $ 0,494  $
	\newpage
	\section*{Задача 6}	
	\subsection*{Условие}

	Найти решение задачи Коши в виде ряда:
	$$ xy'' + (x^2 + 1)y' + 2xy = 10x; \, y(0) = 0, \, y'(0) = 0 $$
	$$ (x^2 + 2) y''(x) + x (y''' + 4 y') + 2 y - 10 = 0 $$
	\subsection*{Решение}

	Найдём решение в виде ряда Маклорена:
	$$ y = \sum_{n=0}^\infty \frac{y^{(n)}(0)}{n!} x^n , \text{ где } y(0) = 0, \,y'(0) = 0, $$
	а остальные значения производных в нуле $ y^{(n)}(0) $, $ n \geq 2 $ последовательно найдём из исходного уравнения:

	\bigskip
	\hspace*{-0.9cm}
	\begin{tabular}{llll}
		$ y'' $			&	$ = -\frac{2 y + x (y''' + 4 y')  - 10}{x^2+2}  $& 	$ \Rightarrow y''(0)     $ 	& 	$ = - \frac{-10}{2} = 5 ,    $ \\
		$ y''' $		&	$ = -\frac{6 y' + x (y^{IV} + 6 y'')}{x^2+3}   $& 	$ \Rightarrow y'''(0)    $ 	& 	$ = - \frac{0}{3} = 0 ,    $ \\
		$ y^{IV} $		&	$ = -\frac{12 y'' + x (y^{V} + 8 y''')}{x^2+4} $& 	$ \Rightarrow y^{IV}(0)  $ 	& 	$ = - 3 \cdot 5 =- 15 , $ \\
		$ \vdots $		&	$      \vdots 								   $& 	$ \vdots				 $ 	& 	$ \vdots                 $ \\
		$ y^{(n+3)} $	&	$ = -\frac{(n+2)(n+3) y^{(n+1)} + x (y^{(n+4)} + 2(n+3) y^{(n+2)}) - 10 x^{(n+2)} }{x^2+n+3} $& 	$ \Rightarrow y^{(n+3)}(0)  $ 	& 	$ = -(n+2) y^{(n+1)} - 10 x^{(n+2)}  .$ \\
	\end{tabular}
	\bigskip

	Первое равенство получили, выразив $ y'' $ из данного в задаче уравнения, предварительно продифференцировав его, для получения второго продифференцировали уравнение второй раз и так далее. Таким образом, получили рекуррентную формулу выражающую значение $ (n+3) $-ей производной в нуле через значение $ n $-ой производной. Поскольку $ y'(0) = y'''(0) = 0 $, то значение всех производных порядка $ 1 + 2m, \, m = 0,1,2..., $ в нуле равны нулю. Отличны от нуля только при $ x = 0 $ только производные, порядок которых четен:
	$$ y^{(VI)}(0) = -5  y^{(IV)}(0)  = 5 \cdot  3 \cdot 5  , \, y^{(VIII)}(0) = -7  y^{(VI)}(0)   = -7 \cdot  5 \cdot  3 \cdot 5 $$
	$$ y^{(2m)}(0) = 5 \cdot  3 \cdot ... \cdot [2m - 1] \cdot (-1)^{m+1}, \, m = 1,2... $$

	\hspace{200pt} \textbf{Ответ:} $y(x) = 5 \sum_{m=1}^\infty \frac{3 \cdot ... \cdot (2m - 1 )\cdot (-1)^{m+1} }{2m!}  x^{2m}$.

\end{document}