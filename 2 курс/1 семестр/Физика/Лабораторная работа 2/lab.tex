%
\documentclass[12pt]{article}

\usepackage{mathtext} 
\usepackage{amsmath}

\usepackage[english, russian]{babel}
\usepackage[TS1]{fontenc}
\usepackage[utf8]{inputenc}
\usepackage{pscyr}
\usepackage[left=2cm,right=2cm, top=1cm,bottom=1.5cm,bindingoffset=0cm]{geometry}

\usepackage{multirow}
\usepackage{hhline}

\usepackage{indentfirst}

\usepackage{enumitem,kantlipsum}

% \usepackage{graphicx}
% \graphicspath{{pictures/}}
% \DeclareGraphicsExtensions{.pdf,.png,.jpg}

% \usepackage{tikz}
% \usetikzlibrary{patterns}
\usepackage{pgfplots}
\pgfplotsset{compat=1.9}
% \usepgfplotslibrary{fillbetween}

% \usepackage{ulem}

% \usepackage{hyperref}  

% \usepackage{circuitikz}

% \usepackage{fp}
% \usepackage{xfp}

% \usepackage{siunitx}
% \sisetup{output-decimal-marker={,}}

% \usepackage{minted}

\let\oldref\ref
\renewcommand{\ref}[1]{(\oldref{#1})}

\begin{document}
    \pagestyle{empty}
    \begin{center}
        \textbf{Федеральное государственное автономное образовательное учреждение высшего образования}
        
        \vspace{5pt}
        
        {\small
            \textbf{САНКТ-ПЕТЕРБУРГСКИЙ НАЦИОНАЛЬНЫЙ ИССЛЕДОВАТЕЛЬСКИЙ  УНИВЕРСИТЕТ ИНФОРМАЦИОННЫХ ТЕХНОЛОГИЙ, МЕХАНИКИ И ОПТИКИ}

            \textbf{ФАКУЛЬТЕТ  ПРОГРАММНОЙ ИНЖЕНЕРИИ И КОМПЬЮТЕРНОЙ ТЕХНИКИ}%
        }

        \vspace{140pt}

        {\Large            
            \textbf{ОТЧЁТ}

            \vspace{7pt}

            \textbf{ПО ЛАБОРАТОРНОЙ РАБОТЕ №2}%
        }

        \vspace{10pt}
        
        {\large
            \textbf{«Определение длины световой волны по картине} 

            \vspace{5pt}

            \textbf{дифракции на круглом отверстии»}%
        }

        \vspace{170pt}
        
        \begin{tabular}{lll}
            Проверил:	 	  							                & \hspace{70pt}	&	Выполнил:							        	\\
            Пшеничнов В.Е.	 \_\_\_\_\_\_\_\_\_\_\_\_\_                 &			    &	Студент группы P3255				        	\\
            «\_\_\_\_\_\_» 	\_\_\_\_\_\_\_\_\_\_\_\_\_\_ \the\year г.	& 			    &	Федюкович С. А. \_\_\_\_\_\_\_\_\_\_\_\_\_\_	\\
			                    							            &			    &									            	\\
                                                                        &			    &										            \\
        \end{tabular}

        \vspace*{\fill}

        Санкт-Петербург

        \the\year
    \end{center}
    \newpage
    \pagestyle{plain}
    \setcounter{page}{1}

    \section*{Цель работы}

    Определение длины световой волны по картине дифракции на круглом отверстии в экране.

    \section*{Теоретические основы}

    При прохождении пучка параллельных лучей света через круглое отверстие в экране свет заходит в область геометрической тени. За экраном наблюдается дифракционная картина в виде чередующихся светлых и тёмных колец. 

	Распределение интенсивности света в дифракционной картине можно рассчитать на основе принципа Гюйгенса-Френеля,  через метод зон Френеля.

    Пусть на экран с круглым отверстием радиусом $ OB $ падает плоская монохроматическая волна. В соответствии с принципом Гюйгенса-Френеля действие этой волны можно заменить действием когерентных точечных источников света. Определим действие этой волны в точке $ P $, лежащей на прямой $ SS' $, проходящей через центр отверстия. Для этого разделим часть волновой поверхности на кольцевые зоны (зоны Френеля), чтобы расстояния от края следующей зоны до точки $ P $ отличались друг от друга на половину длины волны $ \lambda/2 $:
    \begin{equation}
        \label{eq:lam}
        r_1 = r_0 + \frac{\lambda}{2}; \, r_2 = r_1 + \frac{\lambda}{2} = r_0 + 2 \frac{\lambda}{2}; \dots ; r_k = r_0 + k \frac{\lambda}{2}.
    \end{equation}

    При таком делении фазы колебаний, приходящих в точку $ P $ от соседних зон, отличаются на $ \pi $ , т.е. противоположны. Если амплитуды колебаний от $ 1, 2, \dots,k $-ой зон обозначить $ a_1, a_2, \dots , a_k $ , то амплитуда результирующего колебания в точке $ P $:
    \begin{equation}
        \label{eq:a}
        A = a_1 - a_2 + a_3 + \dots + (-1)^{k+1} \cdot a_k
    \end{equation}

    Амплитуда колебаний, приходящих от отдельной зоны, зависит от площади зоны $ \Delta S $, от расстояния $ r_k $ от зоны до точки $ P $ и от угла наклона $ \alpha $  между $ r_k $ и нормалью к поверхности. При при таком способе деления площадь $ k $-ой зоны:
    \begin{equation}
        \label{eq:s}
        S_k = \pi \rho^2_{k+1} - \pi \rho^2_{k} ,
    \end{equation}
    где $ \rho_{k+1} $ и $ \rho_{k} $ --- радиусы $ k $-ой и $ k + 1 $-ой зон. Радиусы зон Френеля определяются соотношениями:
    \begin{equation}
        \label{eq:r}
        \rho^2_{k} = (r_0 + k \frac{\lambda}{2})^2 - r^2_0; \, \rho^2_{k+1} = (r_0 + (k + 1)\frac{\lambda}{2})^2 - r^2_0;
    \end{equation}

    Учитывая, что $ r_0 >> \lambda $, получим $ \rho^2_{k+1} - \rho^2_{k} = r_0 \lambda $ , а площадь $ к $-й зоны $ S_k = \pi r_0 \lambda $ , т.е. площадь зоны Френеля не зависит от номера зоны $ k $. Следовательно, амплитуды колебаний зависят лишь от расстояния $ r $ и от угла $ \alpha $ .  

    Монотонное убывание амплитуд позволяет приближенно выразить амплитуду A суммарного колебания в точке $ P $:
    \begin{equation}
        \label{eq:a2}
        A = \frac{a_1}{2} + \left(\frac{a_1}{2} - a_2 + \frac{a_3}{2} \right) + \left(\frac{a_3}{2} - a_4 + \frac{a_5}{2} \right) + \left(\frac{a_1}{2} - \dots \right)
    \end{equation}

    Так как слагаемые, выделенные скобками, равны нулю, результирующая амплитуда при нечетном $ k : A = \frac{a_1}{2} + \frac{a_k}{2}$  , а при четном $ k : A = \frac{a_1}{2} - \frac{a_k}{2}$  . Объединяя, получаем $ A = \frac{a_1}{2} \pm \frac{a_k}{2} $, где знак $ «+» $ относится к нечетному, а знак $ «-» $ --- к четному числу зон Френеля.

    При свободном распространении, когда не происходит ограничение фронта волны, $ k \to \infty  $ и $ a_k \to \infty $. Тогда при открытом фронте амплитуда суммарного колебания в точке $ P $ определяется половиной амплитуды первой зоны.

    Если отверстие открывает одну зону или их небольшое нечетное число, то в результате интерференции в точке $ P $ будет виден свет, причем более интенсивный, т.е. образуется дифракционный максимум. При небольшом четном числе открытых зон освещенность в точке $ P $ будет минимальной.

    Пусть для точки наблюдения $ P $ открыто $ m $ зон. Тогда при соблюдении предложенного Френелем правила разбиения на зоны, в открытой отверстием части волнового фронта будет умещаться большее число зон. 
    \begin{equation}
        \label{eq:r2}
        R^2  = m \lambda d
    \end{equation}

    Из выражений \ref{eq:r2} расстояние от плоскости отверстия до точки наблюдения:
    \begin{equation}
        \label{eq:d}
        d = \frac{R^2}{m \lambda}
    \end{equation}

    Это соотношение служит для вычисления длины волны. Для повышения точности определения длины волны расстояние $ d $ измеряется несколько раз при разном числе открытых зон $ m $. Как видно из уравнения \ref{eq:d}, зависимость от $ 1/m $ является линейной, а коэффициент наклона графика этой зависимости $ k = R^2 / \lambda $.

    Построив график зависимости $ d $ от $ 1/m $ можно убедиться в том, что зависимость действительно линейна, а по коэффициенту наклона получившейся прямой и известному значению радиуса отверстия R определить длину волны.

    \section*{Ход работы}

    \begin{enumerate}[wide, labelwidth=!, labelindent=0pt]
        \item Установить объектив $ M $ так, чтобы на экране была видна дифракционная картина от отверстия, соответствующая открытым двум зонам Френеля. Записать координату $ X $ по шкале.
        \item Передвигая объектив по направлению к лазеру, наблюдать за сменой освещенности в центре дифракционной картины. Для каждого числа открытых зон записывать координату объектива. Также записать координату выходного окна $ X_\infty $ Заполнить таблицу \ref{tab:1}. Повторить $ 1-3 $ раза. 
        
        \item Определить для каждого $ m $ расстояния $ d $ с учетом того, что $ d = l - b = X - X_\infty $. Результаты добавить в таблицу \ref{tab:1}.
        \begin{table}[h!]
            \caption{Экспериментальные данные}
            \label{tab:1}
            \centering
            \begin{tabular}{|c|c|c|c|c|c|c|c|}
                \hline
                $ m $   &   $ 1/m $ &   $ X_1, [см]$ & $ d_1, [см] $ &   $ X_2, [см] $ & $ d_2, [см] $ &   $ X_3, [см] $ & $ d_3, [см] $ \\
                \hline
                2,000 & 0,500 & 64,000 & 19,400 & 64,700 & 18,700 & 64,500 & 18,900\\ 
 \hline 
3,000 & 0,333 & 70,500 & 12,900 & 70,100 & 13,300 & 70,700 & 12,700\\ 
 \hline 
4,000 & 0,250 & 73,600 & 9,800 & 73,400 & 10,000 & 73,500 & 9,900\\ 
 \hline 
5,000 & 0,200 & 75,700 & 7,700 & 75,600 & 7,800 & 75,700 & 7,700\\ 
 \hline 
6,000 & 0,167 & 77,000 & 6,400 & 77,000 & 6,400 & 77,100 & 6,300\\ 
 \hline 
         
                \multicolumn{8}{|c|}{
                    $X_\infty = 83,400[см];r = 0,500\pm0,020[см]$
                }   \\
                \hline
            \end{tabular}
        \end{table}
        \item Построить график зависимости расстояния $ d $ от $ 1/m $. По коэффициенту наклона $ k $ аппроксимирующей прямой и радиусу отверстия $ r $ определить длину волны источника:
        $$ \text{Уравнение прямой:} \, d = 37,730\cdot 1/m + 0,252
        $$
        \begin{figure}[h!]
            \label{graph:2}
            \caption{График зависимости расстояния $ d $ от $ 1/m $}
            \centering
            \begin{tikzpicture}
				\begin{axis}[		
                    xlabel = {$ 1/m $},
                    ylabel = {$ d,[см] $},	
                    xmin = 0.16,
                    ymin = 6,
                    xmax = 0.51,
                    ymax = 20,	
                    axis x line=center,
                    axis y line=center,
                    width = 500,
                    height = 300,
                    minor x tick num={4},
                    minor y tick num={4},
                    grid = both
				]
				    \addplot[only marks,black,mark size=1pt] coordinates {
                        (0.500,19.000) (0.333,12.967) (0.250,9.900) (0.200,7.733) (0.167,6.367) 
                    };                    
                    \addplot[black, domain=0:0.5] {37.730*x + 0.252};
                    
				\end{axis} 
				\end{tikzpicture}
        \end{figure}
        $$ \lambda = \frac{r^2}{k} = \frac{(0,500)^2}{37,730} \cdot 10^5 = 662,596[нм]
         $$
        \item Рассчитать погрешность наклона $ \Delta k $ и, исходя из нее и погрешности радиуса $ r $ найти погрешность $ \Delta \lambda $:
        \begin{equation*}
            \frac{k}{\Delta k} = \frac{1}{y_b - y_a} \sqrt{\frac{2}{N-2} \cdot \sum_{i=1}^N (y_i - y_{ср})^2 } = 
        \end{equation*}
        \begin{equation*}
            =\frac{1}{0,039 - 0,230} \sqrt{\frac{2}{3} \cdot (0,780^2 + 0,770^2 + 0,290^2 + 0,460^2 + 0,860^2)} = 0,018
        \end{equation*}
        \begin{equation*}
            \Delta k = 0,018 \cdot 37,730 \cdot 10^{-2} = 0,006 []
        \end{equation*}
        \begin{equation*}
            \Delta \lambda = \sqrt{\left(\frac{\partial \lambda}{\partial k} \Delta k \right)^2 + \left(\frac{\partial \lambda}{\partial r} \Delta r \right)^2} = \sqrt{\left(-\frac{r^2}{k^2} \Delta k \right)^2 + \left(\frac{2 r}{k} \Delta r \right)^2} = 
        \end{equation*}
        \begin{equation*}
            =\sqrt{\left(-\frac{0,001}{0,377^2} \cdot 0,006 \right)^2 + \left(\frac{2 \cdot 0,001}{0,377} \cdot 0,001 \right)^2} = 53,3 [нм]
        \end{equation*}
    \end{enumerate}

    \section*{Вывод} 

    В ходе работы я определил длину световой волны по картине дифракции на круглом отверстии на основе принципа Гюйгенса-Френеля,  с помощью метода зон Френеля. 

    Диапазон красного цвета спектра определяют длиной волны $ 620—740  [нм]$, поэтому, значение $ \lambda = 662,596 \pm 53,3 [нм]$  
    полученное в результате выполнения лабораторной работы попадает под заданные значения диапазона.

    Погрешность составила около $ 8,5 \% $, что является приемлемой погрешностью. Вызвана она в связи с неточностью измерений маленьких  величин, а так же погрешностью при расчетах. 
      
\end{document}