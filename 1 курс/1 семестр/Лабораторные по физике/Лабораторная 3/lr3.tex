\documentclass[12pt]{article}
\usepackage{hhline}
\usepackage{graphicx}
\graphicspath{{pictures/}}
\DeclareGraphicsExtensions{.png}
\usepackage{multirow}
\usepackage{amsmath}
\usepackage{mathtext}
\usepackage[T2A]{fontenc}
\usepackage[utf8]{inputenc}
\usepackage{pscyr} 
\usepackage[left=2cm,right=2cm, top=1.5cm,bottom=1cm,bindingoffset=0cm]{geometry}

\begin{document}
\pagestyle{empty}
\begin{center}
\large{\textbf{Университет ИТМО}}
\end{center}
\rule{500pt}{1pt}
\par\bigskip\par\bigskip\par\bigskip\par\bigskip\par\bigskip\par\bigskip\par\bigskip\par\bigskip
\begin{center}
\Large
\textbf{Отчёт по лабораторной работе №3}

\textbf{\textit{«Определение момента инерции крестовины при различном расположении грузов»}}


\end{center}
\par\bigskip\par\bigskip\par\bigskip\par\bigskip\par\bigskip\par\bigskip\par\bigskip\par\bigskip\par\bigskip\par\bigskip\par\bigskip\par\bigskip\par\bigskip\par\bigskip      
\begin{flushright}
\large
Выполнил: Федюкович С. А.
\par\bigskip
Факультет: МТУ “Академия ЛИМТУ”
\par\bigskip
Группа: S3100                       
\par\bigskip\par\bigskip\par\bigskip

\rule{150pt}{0.5pt}
\par\bigskip\par\bigskip\par\bigskip\par\bigskip                                                            
 Проверил: Пшеничников В. Е. 
\par\bigskip \par\bigskip

\rule{150pt}{0.5pt}
\end{flushright}
\par\bigskip\par\bigskip\par\bigskip\par\bigskip\par\bigskip\par\bigskip\par\bigskip\par\bigskip\par\bigskip\par\bigskip     
\begin{center}
\large
Санкт-Петербург
\par\bigskip
2018
\end{center}
\newpage

\section*{Цель работы}
Измерить момент инерции крестовины при заданном расположении грузов на спицах относительно оси вращения.
\section*{Теоретические основы лабораторной работы}
Момент инерции вращающейся системы зависит от распределения массы относительно оси вращения. Эта зависимость имеет вид $I\sim R^2$. В данной работе $R$ --- расстояние от центра груза на спице до оси вращения. Положение груза на первой риске соответсвует $R = 67$ мм, расстояние между рисками $25$ мм.

Основное уравнение динамики вращательного движения в проекции на ось вращения для вращающейся крестовины записывается следующим образом:
\begin{equation}
M_{н}-M_{тр}=I,\varepsilon
\end{equation}
где $M_{н}$ – момент силы натяжения нити, вызывающей вращение; $M_{тр}$ – момент сил трения; $\varepsilon$ – угловое ускорение, $I$ – момент инерции системы.

Вращение крестовины вызвано поступательным движение каретки с шайбами. Это движение описывается следующим  уравнениям динамики: 
\begin{equation}
mg-F_{н}=ma,	
\end{equation}
Здесь $m$ – масса каретки с шайбами, $F_{н}$ – сила натяжения нити. 

Сила натяжения из уравнения $(2)$:
\begin{equation}
F_{н} = mg – ma	
\end{equation}

Считая движение каретки равноускоренным, можно вычислить ускорение по формуле:
\begin{equation}
a = \frac{2h}{t^2}.
\end{equation}

Подстановка выражения $(4)$ в формулу $(3)$ даёт:
\begin{equation}
	F_{н}=m(g-\frac{2h}{t^2 }).	
\end{equation}

Соотвественно момент силы натяжения:
\begin{equation}
	M_{н} = F_{н}r,	
\end{equation}	
Где $r$ – радиус ступицы. 

Выражая радиус ступицы через её диаметр $r=\frac{d}{2}$ и учитывая формулу $(5)$, получаем:
\begin{equation} 
	M_{н}=\frac{md}{2} (g-\frac{2h}{t^2 }).	
\end{equation}

При отсутствии проскальзывания нити, угловое ускорение, с которым вращается система, связано с линейным ускорением через радиус:
	\begin{equation}
	 a=\varepsilon r=\varepsilon \frac{d}{2},
	 \end{equation}
где $d$ – диаметр ступицы, $d = 46,0$ мм.

Объединение формул $(4)$ и $(8)$ даёт расчётную формулу для углового ускорения:
\begin{equation}	
\varepsilon=\frac{4h}{dt^2 }.	
\end{equation}

Из уравнения динамики $(1)$ вращающий момент силы натяжения
\begin{equation}
	M_{н}=M_{тр}+I\varepsilon.
	\end{equation}
 График функции $M_{н}=f(\varepsilon)$ представляет собой прямую линию.
\section*{Экспериментальные данные}
\subsection*{Часть 1}
\begin{table}[h!]
\begin{center}
Таблица 1

\begin{tabular}{|c|c|c|c|c|}
\hline
\multirow{2}{*}{Определяемая величина} & \multicolumn{4}{c|}{Количество шайб на каретке, $k$} \\
\hhline{~----}
 &4  & 3&2&	1	 \\\hline

$t$, с &   4,65	 &9,95	 &11,81	 &14,72\\

\hline
\end{tabular}
\par\bigskip     
Исходные данные:
$H_{0}=0$ м;
$m_{к}=132,0$г;

$h=0,7$м;
$\Delta_{итк}=\Delta_{итш} =1,0$г;
$m_{ш}=200$г. 
\end{center}
\end{table} 
\subsection*{Часть 2}
\begin{table}[h!]
\begin{center}
Таблица 2

\begin{tabular}{|c|c|c|c|c|c|c|}
\hline
\multirow{2}{*}{Определяемая величина} & \multicolumn{6}{c|}{Номер риски на спице} \\
\hhline{~------}
 &1  & 2&3&	4&5&6	
  \\\hline

$t$, с &   6,85&	8.01&	8,98	&9,48	&10,7&	12\\

\hline
\end{tabular}
\end{center}
\end{table} 
\newpage
\section*{Обработка результатов измерений}
\subsection*{Часть 1}
\begin{enumerate}
\item	Рассчитаем массу каретки с шайбами для $k = 1…4$ по формуле: $m=m_{к} = km_{ш}$, где $m_{к}, m_{ш} $--- массы каретки и шайбы, значения которых приведены в таблице 1. Результаты занесём в таблицу 3: 
\begin{table}[h!]
\begin{center}
\begin{tabular}{|c|c|c|c|c|c|c|}
\hline
\multirow{2}{*}{Определяемая величина} & \multicolumn{4}{c|}{Количество шайб на каретке, $k$} \\
\hhline{~----}
 &4  & 3&2&	1	 \\\hline

$m$, г &  932	&732&	532	&332\\
\hline
$\varepsilon, рад/с^2$ & 2,81&	0,61	&0,43&	0,28\\
\hline
$M_{H}, Н\cdot м$ &   0,2	&0,16	&0,12&	0,07\\
\hline
\end{tabular}
\end{center}
\end{table} 
\item Рассчитываем угловое ускорение $\varepsilon$ по формуле $(9)$ и вращающий момент $M_{H}$ по формуле $(7)$. Результаты занесём в таблицу 3.
\item По результатам расчетов строим график зависимости $M_{H}=f(\varepsilon)$; Экстраполируем полученную прямую до пересечения с осью и определяем момент силы трения $M_{тр}$:

\end{enumerate}
\newpage
\subsection*{Часть 2}
\begin{enumerate}
\item Рассчитаем значение $R^2$ и занесём в таблицу 4:
\begin{table}[h!]
\begin{center}
\begin{tabular}{|c|c|c|c|c|c|c|}
\hline
\multirow{2}{*}{Определяемая величина} & \multicolumn{6}{c|}{Номер риски на спице} \\
\hhline{~------}
 &1  & 2&3&	4&5&6	\\
\hline

$R$, м &  0,06&	0,09&	0,12	&0,14	&0,17	&0,19\\
\hline
$R^2, м^2$ &   0,004	&0,008	&0,013&	0,020&	0,027&	0,036\\
\hline
$(M_{H}-M_{тр}), Н\cdot м$ &   \multicolumn{6}{c|}{0,01}\\
\hline
$\varepsilon, рад/с^2$ & 1,297&	0,949	&0,755	&0,677&	0,532&	0,423\\
\hline
$I, кг\cdotм^2 $&  0,008 &	0,010	&0,013	&0,015	&0,018&	0,023\\
\hline
\end{tabular}
\end{center}
\end{table} 

\item Берём из таблицы 3 значение момента $M_{H}$ для опыта с двумя шайбами и с учётом определенного в п.3 первой части обработки результатов измерений момента силы трения $M_{тр}$, записываем в таблицу 4 значение: $M_{H}-M_{тр}\approx 0,01$.
\item Рассчитаем угловое ускорение по формуле $(9)$ и момент инерции по формуле: \\ $I=\frac{M_{H} - M_{тр}}{\varepsilon}$
\item	Построим график зависимости $I=f(R^2)$
\item Экстраполируем полученную прямую до пересечения с осью ординат и определяем момент инерции ступицы со спицами $I_{ст}$. Сравниваем полученное значение с рассчетным 	($I_{ст.расч} = 7\cdot10^{-3}кг\cdotм^2$).
\newpage

\end{enumerate}
\newpage
\section*{Вывод}
Сравнивая полученное значение ($I_{ст}=6\cdot10^{-3}кг\cdotм^2$) с рассчетным ($I_{ст.расч} = 7\cdot10^{-3}кг\cdotм^2$), видим,что разница совсем незначительная. Она вызвана погрешностью в измерении. Кроме того на графике зависимости момента инерции ступицы со спицами от квадрата расстояние груза до оси вращения мы действительно наблюдаем линейную завимость двух величин. Таким образом, экспериментально было доказано, что момент инерции пропорционален квадрату расстояния до центра массы.
\end{document}