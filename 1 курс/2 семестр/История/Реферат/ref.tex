\documentclass[12pt]{article}
\usepackage[russian]{babel}
\usepackage{hhline}
\usepackage{graphicx}
\graphicspath{{pictures/}}
\DeclareGraphicsExtensions{.png}
\usepackage{multirow}
\usepackage{amsmath}
\usepackage{mathtext}
\usepackage[T2A]{fontenc}
\usepackage[utf8]{inputenc}
\usepackage{pscyr} 
\usepackage[left=2.5cm,right=2.5cm, top=1.5cm,bottom=2cm,bindingoffset=0cm]{geometry}
\begin{document}
\pagestyle{empty}

\begin{center}
\large{\textbf{Университет ИТМО}}
\end{center}
\rule{500pt}{1pt}
\par\bigskip\par\bigskip\par\bigskip\par\bigskip\par\bigskip\par\bigskip\par\bigskip\par\bigskip
\begin{center}
\Large
Реферат

Тема:

\textbf{\textit{«Внешняя политика России в 1907 – 1914»}}
\end{center}
\par\bigskip\par\bigskip\par\bigskip\par\bigskip\par\bigskip\par\bigskip\par\bigskip\par\bigskip\par\bigskip\par\bigskip\par\bigskip\par\bigskip\par\bigskip\par\bigskip      
\begin{flushright}
\large
Выполнил: Федюкович С. А.
\par\bigskip
Факультет: МТУ “Академия ЛИМТУ”
\par\bigskip
Группа: S3100                       
\par\bigskip\par\bigskip\par\bigskip

\rule{150pt}{0.5pt}
\par\bigskip\par\bigskip\par\bigskip\par\bigskip                                                            
 Проверил: Шкиро П. И.
\par\bigskip \par\bigskip

\rule{150pt}{0.5pt}
\end{flushright}
\par\bigskip\par\bigskip\par\bigskip\par\bigskip\par\bigskip\par\bigskip\par\bigskip\par\bigskip\par\bigskip\par\bigskip     
\begin{center}
\large
Санкт-Петербург
\par\bigskip
2018
\end{center}
\newpage
\section* {Содержание}

\subsection* {Введение}

\subsection* {1.Основные направления внешней политики в начале XX века.}
\subsubsection* {1.1.Европейское направление. Поиск союзников.}
\subsubsection* {1.2.Дальневосточное направление российской внешней политики.}
\subsubsection* {1.3.Русско-японская война.}

\subsection* {2.Исторический анализ событий и времен начала 20-го века.}
\subsubsection* {2.1.Россия на окраине Европы.}
\subsubsection* {2.2.Специфические черты России.}

\subsection* {Заключение.}

\subsection* {Литература.}
\newpage
\pagestyle{plain}
\setcounter{page}{1}
\section* {Введение}

К началу XX столетия Российская империя была самым большим в территориальном отношении государством мира. Она раскинулась на значительной части Европы и Азии, от Балтийского моря до Тихого океана и от Северного Ледовитого океана до среднеазиатских пустынь. Природа ее отличалась исключительным разнообразием. Хозяйственное развитие различных регионов страны оставалось неравномерным, особо выделялись развивающиеся промышленные регионы: Московский, Петербургский, Рижский, Лодзин-ский, Южно-Российский, Уральский. Начиналось интенсивное освоение Сибири и Дальнего Востока, где центрами становились Красноярск, Новониколаевск (Новосибирск) и Владивосток. Однако огромные пространства были крайне слабо связаны друг с другом транспортными артериями.

Несмотря на отдельные недочеты и ошибки власти, имелись немалые шансы сохранить Российскую империю как целостное государственное образование при одновременном сохранении основ самодержавной монархии на длительный срок. Все недостатки реальной жизни можно было бы ликвидировать или смягчить, реформируя государственный аппарат путем привлечения в него умелых, деятельных администраторов, точно и по-деловому исполняющих монаршью волю.

Индустриальная и рыночная модернизация России призвана была ускорить и политическую модернизацию. Успешное проведение одновременно политических и экономических реформ при уравнении всех подданных государства в правах, независимо от состояния и национальности, создавало определенные опасности для страны, но в то же время и давало немалые возможности для сохранения эволюционного, а не революционного типа развития. В этом варианте развития Российская империя вышла бы на передовые рубежи в разряд крупнейших развитых держав. Но этого не произошло. 
\newpage
\section{Основные направления внешней политики в начале XX века.}
\subsection{Европейское направление. Поиск союзников.}

К началу XX века Российское государство представлялось иностранным наблюдателям как мощная, но слишком изолированная от международных военно-политических интересов сила. На протяжении царствования императора Александра III (1881 — 1894) Россия не вела войн. Такую же политику публично обещал продолжить и император Николай П. Российская дипломатия стала инициатором проведения в 1899 г. международной конференции в Гааге по ограничению вооружений. Однако эта конференция так и не смогла принять каких-либо конструктивных решений. Крупные европейские державы заверяли друг друга в мирных устремлениях, но фактически начали гонку вооружений, завершившуюся через полтора десятилетия мировой войной.

В 1891 г. был подписан русско-французский союз, который был ратифицирован в 1894 г. Стороны обязались оказывать друг другу военную помощь в случае агрессии со стороны стран участниц Тройственного союза. Однако в первое время русско французский союз был направлен не только против Германии, но и против Англии. Сближение с Англией стало возможным лишь в начале ХХ века.

В Европе начали складываться два военно-политических блока: Германии и Австро-Венгрии, ставшего в будущем Тройственным союзом, и России и Франции, превратившемся в Сердечное согласие (Антанту). Европейская роль России в начале XX века усиливалась также и потому, что Российская империя традиционно считала своей обязанностью выступать в защиту народов, относившихся к числу православных: сербов и македонцев, черногорцев и болгар. Более того, во многих странах, населенных славянскими народами, усиливались панславистские настроения. Центром славянского мира идеологи панславизма считали Россию, что давало российской дипломатии возможность активного проведения своей политики в Восточной и Центральной Европе.

Традиционным политическим и военным оппонентом России считалась Османская империя, которая в начале XX века находилась в глубоком кризисе. Ее слабость побуждала многих российских политиков и общественных деятелей ставить вопрос о решении важнейшей, по их мнению, политической задачи: овладения Константинополем (Стамбулом) и превращения черноморских проливов Босфор и Дарданеллы в российскую принадлежность. Закрепить свое влияние в Турции стремились также Великобритания и Германия, что создавало почву для их постоянных трений с Российской империей.

\newpage
\subsection{ Дальневосточное направление российской внешней политики.}
 
Европейское дипломатическое направление не создавало иллюзий быстрого успеха, а мечты о приобретении черноморских проливов излагались скорее в теоретической плоскости. На фоне этого дальневосточное направление российской внешней политики стало в первые годы XX века самым привлекательным. Здесь, на Дальнем Востоке, сконцентрировались дипломатические, военные и экономические интересы ряда государств.

Наиболее радикальную позицию в вопросе быстрого продвижения России на Дальний Восток занимала группа лиц из высшего света, которую возглавлял отставной офицер Кавалергардского полка Безобразов. Они имели личные экономические интересы, создав компанию по эксплуатации природных богатств Кореи. Данная группа получила название «безобразовской шайки». «Безобразовцы» требовали немедленного присоединения Маньчжурии к России.

Однако и относительно осторожная политика Витте и откровенно агрессивная политика «безобразовцев» не учитывала ряд объективных факторов. Во-первых, явно завышалась реальная экономическая мощь России. Сил у империи было недостаточно. Во-вторых, была недооценена активность Японии как основного соперника России в этом регионе. Япония соглашалась только лишь на признание «железнодорожных» интересов России в Маньчжурии, требуя в то же время полной свободы для себя. В третьих, не были верно учтены интересы в Китае таких стран, как США и Англия, поддерживавших Японию. Союзник России — Франция заявила о своем нейтралитете в русско-японских противоречиях. Неожиданно поддержку России оказала Германия. Но и это было объяснимо: германская дипломатия была заинтересована в том, чтобы Россия как можно глубже увязла на Дальнем Востоке и не препятствовала экспансионистским планам Германии уже в самой Европе. Так к началу 1904 г. Россия оказалась в дипломатической изоляции.

Следует учесть, что весь комплекс российской политики, названный «большой азиатской программой», не встречал сочувственного отклика среди значительной части образованного общества. Явно или полускрыто внешняя политика правительства подвергалась критике в самых различных кругах. В свою очередь общественность и публицистика европейских стран и США, заинтересованных в ослаблении влияния России на Дальнем Востоке, постоянно писали об «особой агрессивности» России. Тем не менее неопровержимым историческим фактом является то, что 27 января 1904 г. именно Япония стала агрессором. Почти за неделю до этого российское правительство направило правительству Японии послание, в котором шло на важные уступки Японии, настаивая лишь на том, чтобы Япония не использовала Корею в «стратегических интересах». Но Япония специально задержала передачу этого послания в русское посольство в Токио. Правительство Японии, сославшись на «медлительность» России, разорвало с ней дипломатические отношения, а японская эскадра без объявления атаковала русские корабли на рейде Порт-Артура. Началась русско-японская война.

\newpage
\subsection{Русско-японская война. }

Инициатива в развязывании войны с Россией принадлежала Японии. Однако этой войны объективно желала и часть высших сановников России. За несколько месяцев до ее начала министр внутренних дел В. К. Плеве говорил   о   желательности   «маленькой   победоносной войны» для укрепления авторитета власти. 

Тихоокеанский флот возглавил способный адмирал С. Макаров, надеявшийся проводить активные операции на море. За два первых месяца борьбы на море потери русского флота оказались слишком велики. Во время атаки 27 января 1904 г. у Владивостока были нанесены серьезные повреждения двум российским броненосцам и одному крейсеру. На рейде корейского порта Чемульпо в неравном бою были затоплены командами, во избежание сдачи, крейсер «Варяг» и канонерская лодка «Кореец». С. Макаров погиб на броненосец «Петропавловск», ушедший на дно через две минуты после того, как подорвался на японской мине. Потери росли. Оставшиеся корабли не могли организовать серьезного сопротивления японскому флоту.

Власти, казалось бы, сделали соответствующие выводы. На пост министра внутренних дел был назначен  более  умеренный  по  сравнению  со  своим предшественником В. К. Плеве князь П. Д. Свято-полк-Мирский. Он высказал свое расположение к земскому   движению.   В   конце   1904   —   начале 1905  г.  либеральная интеллигенция предприняла оригинальный политический шаг, названный «банкетной кампанией». На банкетах, происходивших в честь  сорокалетия  судебной реформы,  произносились тосты и речи в пользу более широкого народного представительства. Осенью 1904 г. в Париже состоялся съезд «оппозиционных и революционных партий» с участием как либералов различных оттенков, так и социалистов-пораженцев.    Участник    этого    съезда П. Н. Милюков впоследствии вспоминал, что либеральная часть съезда не была осведомлена о том, что социалисты,   предлагавшие   устроить   вооруженное восстание в Петербурге,  получили на организацию своих действий деньги от японских агентов. Однако окончательно этот факт по сей день не доказан. Тем не менее «революционная» часть съезда на отдельном секретном заседании наметила программу действий на  1905 г.,  включающую активные выступления, вплоть до применения террора.

И, наконец в Цусимском проливе была разгромлена 2-я Тихоокеанская эскадра, шедшая с октября 1904 г. из Балтийского моря. Цусима стала символом поражения России в этой войне.

Поражения под Порт-Артуром, Мукденом, Цусимой усиливали массовое недовольство. Известия о них приходили в страну, которая уже отличалась от той, что вступила в войну в начале 1904 г.  Но и Япония уже была на грани истощения, фактически израсходовав свои резервы. Посреднические услуги в ведении переговоров предложил президент США Теодор Рузвельт. Переговоры проходили в американском городе Портсмуте. Российскую делегацию возглавлял С. Ю. Витте. Японцы выдвинули тяжелые условия, включая выплату контрибуции, передачу Японии всего Сахалина, ограничения числа судов Тихоокеанского флота и т. п. Витте повел себя очень жестко, согласившись фактически лишь на передачу южной части Сахалина. Неожиданно для всех, даже для самого Витте, японцы приняли его план. Николай II даровал ему титул графа, но достаточно скоро, при первом удобном случае отправил в отставку.

Война закончилась, но она фактически стала катализатором оппозиционных настроений, консолидировала различные круги оппозиции, создала благоприятный психологический фон для нарастания революционных событий.
\newpage

\section{ Исторический анализ событий и времен начала XX века.}

\subsection{ Россия на окраине Европы}

В России того времени возможности для быстрого экономического развития и преобразования, которые особенно проявились в периоды промышленных рывков между 1892 - 1899 гг. и 1909 - 1913 гг., были в целом лучше, чем в современных \rq\rq{}развивающихся странах\rq\rq{}. Существует точка зрения, в соответствии с которой сами размеры страны могут также являться преимуществом, способствующим быстрому экономическому развитию. Количество населения как потенциальный потребительский рынок, огромная территория России и ее природные богатства в соответствии с этой точкой зрения должны были способствовать экономическому росту. Азиатская часть России могла играть роль одновременно Британской Индии и американского Дикого Запада.

Однако было мало шансов, что эти благоприятные экономические условия в России сохранятся надолго. Даже в 1913 г. 67\% объема экспорта в стоимостном выражении составляло сельскохозяйственное сырье, а практически все остальное - полезные ископаемые. Однако после первой мировой войны условия внешней торговли для сырья и в особенности для пищевых продуктов стали ухудшаться. Основной фактор, обеспечивающий российский активный платежный баланс, и "двигатель" внутреннего рынка России подошел к точке, с которой начинался долговременный спад.

Второй источник \rq\rq{}активного платежного баланса\rq\rq{}, капиталовложений и экономического развития был внешним (т.е. определялся политикой поощрения иностранных инвестиций и резкого увеличения внешнего долга правительства). Многие считали, что без притока иностранного капитала быстрое развитие российской промышленности будет вовсе невозможно. По существующим оценкам, иностранные вложения за период 1898 - 1913 гг. составили 4225 млн. рублей, из которых около 2000 млн. рублей составляли государственные займы. Влияние иностранного капитала росло. В частности, в то время как за период с 1881 по 1913 г. около 3000 млн. рублей были вывезены из России в качестве доходов с иностранного капитала, крупные средства были реинвестированы. К 1914 г. в России было 8000 млн. рублей иностранных инвестиций. Сюда входят две трети российских частных банков, принадлежавших иностранному капиталу, а также значительное количество шахт и крупных частных промышленных предприятий. Вот как одно поколение спустя Мирский обобщил фактические и потенциальные результаты этого процесса: \rq\rq{}К 1914 г. Россия проделала значительный путь в сторону того, чтобы стать полуколониальным владением европейского капитала\rq\rq{}. Уже к 1916 г. военные расходы более чем удвоили внешний долг, и это было только начало. Кроме того, война значительно усугубила технологическую зависимость России от ее западных союзников. Если бы ей \rq\rq{}не помешали\rq\rq{} (мы снова используем слова Тимашева, говорящего об экстраполяции той же линии развития), Россия после первой мировой войны столкнулась бы с крупнейшим и разрастающимся кризисом погашения внешнего долга и дальнейших займов, чтобы выплатить старые долги, дивиденды и оплатить иностранные патенты и импортные поставки. Подобный сценарий нам хорошо известен на примере современной Латинской Америки, Африки и Азии, будь то Бразилия, Нигерия или Индонезия.

\newpage
\subsection{ Специфические черты России.}

Для царской России вряд ли можно было говорить однозначно и о железном законе спада или же об очевидности продолжения экономического бума 1909 - 1913 гг, т.е., выражаясь языком нашего поколения, о варианте теории зависимого развития или о модернизации. Россия в своем социально-экономическом развитии пыталась угнаться за временем, и никто не мог сказать, каков будет финал этой гонки - и это не просто риторическая фраза или эклектический отказ \rq\rq{}подставиться\rq\rq{}, дав четкий определенный ответ. Это подтверждается статистическими данными. Цифры свидетельствуют, что на всем протяжении рассматриваемого периода Россия ни догоняла, ни отставала все более от своих западных соперников. Между 1861 и 1913 гг. темпы роста национального дохода на душу населения в России приблизительно соответствовали средним европейским показателям и были в два раза медленнее, чем в Германии. Российские показатели роста национального дохода были выше, чем средние показатели неевропейских стран, однако значительно ниже, чем в США и в Японии. Ожидалось дальнейшее ухудшение шансов России в этой гонке, что придавало фактору времени особую важность. В подобной ситуации также имеет значение не только матрица причин, тенденций и объективных факторов, но и фактор сознания, т.е. активный поиск альтернатив властями, силы, на которые они могли рассчитывать, задачи, которые перед ними стояли, и то, каким образом эти задачи понимались и решались.

Первый проблеск нового прагматического понимания этих вопросов появился в среде правящих элит Германии, Японии и России. К тому времени обозначилась третья промежуточная группа стран, помимо удачливых \rq\rq{}призеров\rq\rq{}(т.е. тех стран, которые воспользовались плодами раннего развития торгового, промышленного и колониального капитализма) и прочих (часто колонизированных) народов. Эта третья группа состояла из стран, которые достигли порога широкомасштабной индустриализации несколько позднее, чем "призеры", но чья экономика не была искажена недавним иностранным завоеванием и колониализмом - прямым или косвенным.

Этот список возглавляли США за исключением южных районов, где экономика основывалась на рабовладении и выращивании хлопка, в этой стране не было сильных и укорененных докапиталистических классов, институтов и традиций. Она была достаточно удалена от Европы, чтобы уберечься от политических противоречий и войн, и в то же время находилась достаточно близко, чтобы пользоваться ее рынками, рабочей силой и опытом. Своим \rq\rq{}ростом\rq\rq{} она во многом была обязана труду независимых мелких фермеров на \rq\rq{}открытых границах\rq\rq{} (т.е. на землях, населенных малочисленными народами, которые можно было победить, запереть в \rq\rq{}резервациях\rq\rq{} или истребить). Штатам также благоприятствовало ослабление британского, французского и немецкого контроля, проявившееся в схватке за мировое господство во время первой мировой войны.

Костяк третьей группы стран составляли Германия, Япония и Россия, причем Россия обычно стояла здесь на последнем месте по своим социоэкономическим и политическим показателям и достижениям. Несмотря на многие различия, касающиеся прошлого и настоящего этих стран, все они имели ярко выраженные сходства в правительственной политике и идеологии. Их политика и идеология определялись стремлением избежать \rq\rq{}зависимости\rq\rq{} (как бы мы это назвали сегодня) и \rq\rq{}аккумуляции недостатков\rq\rq{} путем мощного государственного вмешательства, направленного на обеспечение быстрой индустриализации. Это предполагает сильное, активное и диктаторское правительство, которое бы успешно противодействовало внешним нажимам и в то же время контролировало бы \rq\rq{}внутренние политические проблемы\rq\rq{}, будь то социалистическая агитация, требования этнических \rq\rq{}меньшинств\rq\rq{} или даже реакционные выступления со стороны \rq\rq{}правящего класса\rq\rq{} землевладельцев. Цель была одна: прогресс, - \rq\rq{}не мытьем, так катаньем\rq\rq{}, модернизируя армию, способствуя накоплению капитала, индустриализации, отодвигая сельское хозяйство большинства населения на вторые роли в экономике страны.

На протяжении трех десятилетий российское правительство упрямо следовало \rq\rq{}германским путем\rq\rq{}. Бунге, Вышнеградский, Витте, Коковцев, сменявшие друг друга на посту министра финансов, проводили политику \rq\rq{}направляемого\rq\rq{} экономического развития и активного государственного вмешательства, в рамках которого центральная роль отводилась всемерной поддержке национальной промышленности. Правительственная политика способствовала извлечению высоких доходов промышленниками, сохранению низкой заработной платы рабочих и выжиманию соков из крестьянской экономики путем поддержания разрыва цен промышленного и сельского производств ради накопления городского капитала.

Однако, несмотря на все усилия, наличие примера для подражания и амбиции, за Германией Россия угнаться не смогла. Вначале это проявилось в международных политических и финансовых конфликтах. Из ведущей мировой державы первой половины XIX в. Россия к концу века превратилась в государство второй категории. Военное поражение от Японии в 1904 г. и дипломатическое отступление под давлением Австро-Венгрии на Балканах в 1908 г. Все эти удары свидетельствовали о растущей слабости России на международной арене и ослабляли ее еще больше. В то же время жестокий экономический кризис, потрясший Россию на рубеже веков, показал, насколько неустойчивым был ее экономический рост. К внутренним проблемам России добавлялись еще социальные и этнические противоречия и революционный напор. Таким образом, в ситуации, когда нарастал политический и экономический кризис и ослаблялись позиции российского самодержавия на международной арене и внутри страны, политические проекты Витте превратить Россию во вторую Германию вряд ли были достаточно обоснованными.

Как это часто случается, новая драма разыгрывалась в старых терминологических костюмах. Кроме того, новое понимание проявлялось в основном в политических стратегиях и решениях, а не в академических трактатах. Несмотря на это, смысл этого нового понимания был ясен, а также осознавалась его новизна. Пока теория плелась где-то сзади, фактические правители России начали осознавать, что теория, основанная на \rq\rq{}классическом капитализме\rq\rq{}, даже кое-как приспособленная к условиям России, не годится для того типа общества, которым была или становилась Россия.

Поэтому не случайно, что, в то время как многочисленные \rq\rq{}западные\rq\rq{} интеллектуальные моды приходят и уходят, аналитические взгляды, выражающие российский опыт начала века, сохраняют удивительную стойкость, когда речь идет о вопросах \rq\rq{}экономического роста\rq\rq{} и о социальных группах в \rq\rq{}развивающихся обществах\rq\rq{}. Поэтому также слова Витте и Ленина, Столыпина и Сталина звучат и сегодня так, словно они обращены к нынешним политикам и борцам по разные стороны идеологических баррикад в \rq\rq{}развивающихся странах\rq\rq{} во всем мире. В значительной степени этими людьми и был представлен практически весь поныне существующий спектр альтернативных стратегий, теоретических и практических (разве что сюда стоило бы добавить Мао?).

Итак, специфические черты России как \rq\rq{}развивающегося общества\rq\rq{} обусловили значительное отличие ее социальной структуры от других \rq\rq{}догоняющих\rq\rq{} стран в процессе индустриализации.
\section* {Заключение}

Вопросы развития России на рубеже XIX – XX веков всегда вызывали интерес как отечественных так и зарубежных историков. Попытки выделить специфические черты, определить, возможен ли был другой путь развития могучего государства, могла ли Россия выйти из «исторических бурь» начала XX века с меньшими потерями, предпринимались на протяжении всего XX века.   

В советской науке решались схожие вопросы и велись подобные дискуссии, которые строились в основном вокруг вопросов об иностранном капитале и его роли, о действительной степени экономического прогресса в предреволюционной России, о сохранившихся \rq\rq{}феодальных пережитках\rq\rq{} и т.д. В области аграрной истории эти споры разворачивались особенно полно, что объясняет ее важность в рамках общей дискуссии и в академических столкновениях прошлого, настоящего и, несомненно, будущего. Никто не попытался применить модель \rq\rq{}развивающихся обществ\rq\rq{}, чтобы предложить альтернативу однолинейному объяснению, однако все, кто подчеркивал своеобразие социальных преобразований в российской деревне, \rq\rq{}полуфеодализм\rq\rq{} или \rq\rq{}особенности эпохи империализма\rq\rq{}, в сущности, выражал ту же идею. Ленинское любимое ругательство "азиатчина" в применении к России никогда не было должным образом оценено по существу, однако многократно повторялось советскими учеными, чтобы подчеркнуть своеобразие российского капитализма, его "половинчатую" природу, т.е. не вполне капиталистическую и не вполне западноевропейскую. Фундаментальные различия и споры по существу часто скрывались за количественными определениями, т.е. для кого-то капитализм был очень "полу", для кого-то менее "полу" и совсем не "полу" для тех, кого уже Маркс назвал "русскими поклонниками капиталистической системы" (т.е. российских последовательных эволюционистов).

Отголоски принципиального несогласия среди советских историков слышны были также в обсуждении вопроса об "империалистической стадии капитализма" в России.

Очевидно, что все эти общие проблемы не могут быть разрешены путем простого накопления фактов, архивных документов или цифр. Невозможно подвергать сомнению важность тщательного изучения фактического материала, однако нам представляется, что прояснению этих проблем мешают, прежде всего, недостатки концептуальных исторических схем.
\section* {Литература}
\begin {enumerate}

\item Кристофер Мартин. Русско-японская война. 1904-1905. М.: Центрполиграф, 2003.

\itemИстория дипломатии. - 2 изд., Т. 2, М., 1963. С.98.

\itemНовейшая история Отечества: XX век: Учеб. для студ. высш. учеб. заведений: В 2 т./Под ред. А.Ф.Киселева, Э.М.Щагина. - 2-е изд., испр. и доп. - М.: Гуманит. изд. центр ВЛАДОС, 2002. - Т. 1.

\itemРусско-японская война 1904-1905. Взгляд через столетие. Международный исторический сборник под ред. О.Р. Айрапетова. - М.: Модест Колеров и издательство "Три квадрата", 2004, 656 с., 16 с. илл.

\itemРоссия и СССР в войнах XX века. Потери вооруженных сил. Статистическое исследование. Под общ. ред Г. Ф. Кривошеева. М.: “Олма-Пресс” 2001.
\end {enumerate}
\end{document}